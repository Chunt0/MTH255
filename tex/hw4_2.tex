\documentclass[11pt]{article}
\usepackage[margin=.9in]{geometry}
\title{Problem 2}
\author{Christopher Hunt}
\date{}
\usepackage{graphicx} 
\usepackage{fancyhdr}

\begin{document}
\pagestyle{fancy}
\fancyhf{}
\rfoot{MTH 255}
\lfoot{Christopher Hunt}
\lhead{Problem 2}
\rhead{\thepage}
\maketitle

\section*{Problem 2}
\subsection*{(a)}
Newton's Law of Gravitation States that the gravitational field exerted by an object at the origin on another object is given by:
\begin{equation}
    \vec{F} = -G \frac{m_1m_2}{r^2}\hat{r}\;N
\end{equation}
where r is the distance between the two masses.
\\
\\
Find the work done by the gravitational field when the earth moves from aphelion (at a maximum distance of $1.52*10^{11}$ m from the sun) to perihelion (at a minimum distance of $1.47*10^{11}$ m).
$$G = 6.67 * 10^{−11} \; \frac{Nm^2}{kg^2}\qquad m_{Earth} = 5.97 * 10^{24} \;kg \qquad m_{Sun} = 1.99 * 10^{30}\;kg$$
$$a = 1.52*10^{11}\;m \qquad b = 1.47*10^{11}\;m$$

Since these values are all constant we will equate them to a constant k:

$$k = G*m_{Earth}*m_{Sun}$$

Now the function $\vec{F}$ can be written as:

$$\vec{F} = -k*\frac{1}{r^2}\hat{r}\;N$$

So attempting a line integral between the dot product of this vector field and the elliptical path, $1 = \frac{x^2}{a^2}+\frac{y^2}{b^2}$, might be very time consuming, so let's try to find a potential function.
\\
\\
The full form of $\vec{F}$ is:
$$\vec{F} = -k*\frac{1}{r^2}\hat{r} + 0\hat{\phi}+0\hat{\theta}$$

$$f(r,\phi,\theta) = -k \int \frac{1}{r^2} \; dr \rightarrow f(r,\phi,\theta) = -k \left( -\frac{1}{r}+c \right) \rightarrow f(r,\phi,\theta) = \frac{k}{r}-kc$$

Since we found a potential function and it does not depend on $\phi$ or $\theta$ we can run with it! This vector field is conservative, so it is path independent.
\\
\\
To find the work done all that we need to do is find the difference when we plug in a and b to this potential function.

$$W= f(b,0,0) - f(a,0,0) = 1.77 * 10^{32} J$$

Earth experiences $1.77*10^{32}\;J$ of work from the gravitational field of the sun.


\subsection*{(b)}
According to Coulomb’s law the electric force exerted by a charge $q_1$ located at the origin on a charge $q_2$ located at $P_1 = (x_1, y_1, z_1)$ is given by:
\begin{equation}
    \vec{F} = \frac{\epsilon q_1 q_2}{r^2}\hat{r}
\end{equation}
Show that if $q_1$ is a unit charge (equals 1), the work done by $\vec{F}$ to move the charge $q_2$ from $P_1$ to $P_2 = (x_2, y_2, z_2)$ is equal to:
$$\frac{\epsilon q_2}{r_1} - \frac{\epsilon q_2}{r_2}$$
where $r_1 = \sqrt{x_1^2+y_1^2+z_1^2}$ and $r_2= \sqrt{x_2^2+y_2^2+z_2^2}$
\\

This Vector Field is nearly identical to problem 2a. We will take the same approach as above. Since $q_1=1$ we can rewrite the vector field as:
$$\vec{F} = \frac{\epsilon q_1 q_2}{r^2}\hat{r} \rightarrow \vec{F} = k*\frac{1}{r^2} \hat{r} \quad where\;\; k = \epsilon*1*q_2$$
The full form of $\vec{F}$ is:
$$\vec{F} = k*\frac{1}{r^2}\hat{r} + 0\hat{\phi}+0\hat{\theta}$$

$$f(r,\phi,\theta) = k \int \frac{1}{r^2} \; dr \rightarrow f(r,\phi,\theta) = k \left( -\frac{1}{r}+c \right) \rightarrow f(r,\phi,\theta) = -\frac{k}{r}+kc$$

To find the work done all that we need to do is find the difference when we plug in a and b to this potential function.

$$W= f(r_2,0,0) - f(r_1,0,0) = \frac{-k}{r_2} - \frac{-k}{r_1} = \frac{k}{r_1}-\frac{k}{r_2} $$

By using what we know about conservative vector fields and their corresponding potential functions we are able to show that $\frac{\epsilon q_2}{r_1} - \frac{\epsilon q_2}{r_2}$ is true for this scenario.

\end{document}
