\documentclass[11pt]{article}
\usepackage[margin=.9in]{geometry}
\title{Fluid Flow}
\author{Christopher Hunt}
\date{}
\usepackage{graphicx} 
\usepackage{fancyhdr}

\begin{document}
\pagestyle{fancy}
\fancyhf{}
\rfoot{MTH 255}
\lfoot{Christopher Hunt}
\lhead{Fluid Flow}
\rhead{\thepage}
\maketitle

A fluid is flowing along a cylindrical pipe of radius a in the $\hat{i}$ direction. The velocity of the fluid at a radial distance r from the center of the pipe is $\vec{v} = u(1 - \frac{r^2}{a^2})\hat{i} \frac{cm}{sec}$.

\section*{(a)}

When $r = a$ the velocity is $0\frac{cm}{sec}$. The value $u$ is the max velocity of the fluid in the pipe which is reached when $r = 0$, at the center of the pipe.

\section*{(b)}
To find the flux through a circular cross-section, begin by finding the surface differential:
$$x = 0 \quad 0 \leq r \leq a \quad 0 \leq \phi \leq 2\pi$$
$$d\vec{r} = dr \hat{r} + rd\phi\hat{\phi} + dx \hat{i}$$
$$d\vec{r_1} = dr\hat{r} \qquad d\vec{r_2} = rd\phi \hat{\phi} $$
$$d\vec{A} = d\vec{r_1} \times d\vec{r_2} = rdrd\phi\hat{i}$$

Now find the Flux:
$$Flux = \int_C \vec{G} \cdot d\vec{A} \rightarrow \int_0^{2\pi}\int_0^{a} (ur - \frac{ur^3}{a^2})drd\phi$$

Computing this double integral gives us:
$$Flux = \frac{ua^2\pi}{2} \; \frac{cm^3}{sec}$$

\end{document}

