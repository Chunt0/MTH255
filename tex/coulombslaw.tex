\documentclass[11pt]{article}
\usepackage[margin=.9in]{geometry}
\usepackage{xcolor}
\title{Coulomb's Law}
\author{Christopher Hunt}
\date{}
\usepackage{graphicx} 
\usepackage{fancyhdr}

\begin{document}
\pagestyle{fancy}
\fancyhf{}
\rfoot{MTH 255}
\lfoot{Christopher Hunt}
\lhead{Coulomb's Law}
\rhead{\thepage}
\maketitle

\subsection*{Given the electrostatic field $\vec{E}$ at the point $\vec{r}$ due to a charge $q$ at $P(0,0,0)$ is:}
$$ \vec{E}(\vec{r}) = \frac{q}{4\pi \epsilon_o}*\frac{\vec{r}}{|\vec{r}|^3}$$

\subsection*{a) Compute $div(\vec{E})$:}

Using spherical components:
$$\vec{E} = \frac{q}{4\pi \epsilon_o}*\frac{1}{r^2}\; \hat{r}$$
$$div\vec{E} = \frac{1}{r^2}\frac{\delta}{\delta r}(r^2 \vec{E}_{\hat{r}}$$
$$div(\vec{E}) = 0$$

\subsection*{b) Let $S_a$ be the sphere of radius a.}

The electrostatic flux through the sphere of radius a is:
$$\Phi_{S_a} = \oint_S \vec{E} \cdot d\vec{A} = \frac{q}{\epsilon_0}\frac{N m^2}{C}$$

By leveraging what we know about the flux through a sphere around a point charge from Guass's Law, the flux through any surface can be calculated using the divergence theorem if the surface area that the flux is being calculated through has a hollow sphere in the center which surrounds the point charge.

\subsection*{c) Consider an outward oriented arbitrary surface, $S$ around a point charge q.}
\begin{center}
    \includegraphics[scale=.3]{spherecube.png}
\end{center}

To find the flux through the cube, $S$ we will place a sphere, $S_a$ within it to form a new surface $S_b$. From this we can formulate the following equality:

$$\Phi_{S} - \Phi_{S_a} = div(\vec{E}) * dV_{S_b} = 0 $$
$$\Phi_S - \frac{q}{\epsilon_0} = 0 \rightarrow \Phi_S = \frac{q}{\epsilon_0}$$

By using the divergence theorem, Gauss's Law is proven once again. For any arbitrary surface $S$ enclosing a point charge the Electric Flux through that surface will be:
$$\Phi_{S} = \frac{q}{\epsilon_0}\frac{N m^2}{C}$$

\end{document}

